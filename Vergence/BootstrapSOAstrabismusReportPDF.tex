\documentclass[]{article}
\usepackage{lmodern}
\usepackage{amssymb,amsmath}
\usepackage{ifxetex,ifluatex}
\usepackage{fixltx2e} % provides \textsubscript
\ifnum 0\ifxetex 1\fi\ifluatex 1\fi=0 % if pdftex
  \usepackage[T1]{fontenc}
  \usepackage[utf8]{inputenc}
\else % if luatex or xelatex
  \ifxetex
    \usepackage{mathspec}
  \else
    \usepackage{fontspec}
  \fi
  \defaultfontfeatures{Ligatures=TeX,Scale=MatchLowercase}
\fi
% use upquote if available, for straight quotes in verbatim environments
\IfFileExists{upquote.sty}{\usepackage{upquote}}{}
% use microtype if available
\IfFileExists{microtype.sty}{%
\usepackage{microtype}
\UseMicrotypeSet[protrusion]{basicmath} % disable protrusion for tt fonts
}{}
\usepackage[margin=1in]{geometry}
\usepackage{hyperref}
\hypersetup{unicode=true,
            pdftitle={Bootstrapping SOA in Strabismus},
            pdfborder={0 0 0},
            breaklinks=true}
\urlstyle{same}  % don't use monospace font for urls
\usepackage{graphicx,grffile}
\makeatletter
\def\maxwidth{\ifdim\Gin@nat@width>\linewidth\linewidth\else\Gin@nat@width\fi}
\def\maxheight{\ifdim\Gin@nat@height>\textheight\textheight\else\Gin@nat@height\fi}
\makeatother
% Scale images if necessary, so that they will not overflow the page
% margins by default, and it is still possible to overwrite the defaults
% using explicit options in \includegraphics[width, height, ...]{}
\setkeys{Gin}{width=\maxwidth,height=\maxheight,keepaspectratio}
\IfFileExists{parskip.sty}{%
\usepackage{parskip}
}{% else
\setlength{\parindent}{0pt}
\setlength{\parskip}{6pt plus 2pt minus 1pt}
}
\setlength{\emergencystretch}{3em}  % prevent overfull lines
\providecommand{\tightlist}{%
  \setlength{\itemsep}{0pt}\setlength{\parskip}{0pt}}
\setcounter{secnumdepth}{0}
% Redefines (sub)paragraphs to behave more like sections
\ifx\paragraph\undefined\else
\let\oldparagraph\paragraph
\renewcommand{\paragraph}[1]{\oldparagraph{#1}\mbox{}}
\fi
\ifx\subparagraph\undefined\else
\let\oldsubparagraph\subparagraph
\renewcommand{\subparagraph}[1]{\oldsubparagraph{#1}\mbox{}}
\fi

%%% Use protect on footnotes to avoid problems with footnotes in titles
\let\rmarkdownfootnote\footnote%
\def\footnote{\protect\rmarkdownfootnote}

%%% Change title format to be more compact
\usepackage{titling}

% Create subtitle command for use in maketitle
\newcommand{\subtitle}[1]{
  \posttitle{
    \begin{center}\large#1\end{center}
    }
}

\setlength{\droptitle}{-2em}
  \title{Bootstrapping SOA in Strabismus}
  \pretitle{\vspace{\droptitle}\centering\huge}
  \posttitle{\par}
  \author{}
  \preauthor{}\postauthor{}
  \date{}
  \predate{}\postdate{}


\begin{document}
\maketitle

\section{Introduction}\label{introduction}

We are trying to determine if SOA cells in strabismus are sensitive to
vergence velocity. We fit a linear model that predicts firing rate based
on the vergence angle, the vergence velocity and the vertical position
of the eyes. The vertical position term was included to account for
pattern strabismus, which is not encoded by SOA neurons according to
previous work.

\section{Results}\label{results}

\subsection{Slow Vergence Movements}\label{slow-vergence-movements}

\includegraphics{BootstrapSOAstrabismusReportPDF_files/figure-latex/slow-1.pdf}
\includegraphics{BootstrapSOAstrabismusReportPDF_files/figure-latex/slow-2.pdf}
\includegraphics{BootstrapSOAstrabismusReportPDF_files/figure-latex/slow-3.pdf}

\subsubsection{Significance Summary}\label{significance-summary}

We analyzed 58 cells and found that 19 (33\%) had significant
sensitivity to both vergence velocity and vergence position. A total of
34 (59\%) cells showed significant sensitivity to vergence angle during
slow vergence, while 29 cells showed significant sensitivity to vergence
velocity. It seems likely that we would get a better estimate of
position sensitivity by analyzing the firing rates of neurons during
fixation.

There is a huge outlier in QT-103. It has very strong sensitivity to
vergence angle, but a negative sensitivity to vergence velocity. I'm not
sure what this means or if it is a real effect. It would suggest that
the cell pauses during convergence movements but fires more when the
eyes are converged, which is strange behavior.
QT\_2012\_08\_21\_1104\_Radial\_Dump\_AllSOAanalysis\_EyeSpec.csv

\subsection{Saccades with either eye viewing (no eye switching) includes
normal}\label{saccades-with-either-eye-viewing-no-eye-switching-includes-normal}

\includegraphics{BootstrapSOAstrabismusReportPDF_files/figure-latex/saccades both and normal-1.pdf}
\includegraphics{BootstrapSOAstrabismusReportPDF_files/figure-latex/saccades both and normal-2.pdf}
\includegraphics{BootstrapSOAstrabismusReportPDF_files/figure-latex/saccades both and normal-3.pdf}

\subsection{Saccades with either eye viewing (no eye
switching)}\label{saccades-with-either-eye-viewing-no-eye-switching}

\includegraphics{BootstrapSOAstrabismusReportPDF_files/figure-latex/saccades both-1.pdf}
\includegraphics{BootstrapSOAstrabismusReportPDF_files/figure-latex/saccades both-2.pdf}
\includegraphics{BootstrapSOAstrabismusReportPDF_files/figure-latex/saccades both-3.pdf}

\subsubsection{Significance Summary}\label{significance-summary-1}

We analyzed 62 cells that had sufficiently many saccades without eye
switches and found that 11 (18\%) had significant sensitivity to both
vergence velocity and vergence position. A total of 32 (52\%) cells
showed significant sensitivity to vergence angle during these saccades,
while 16 cells showed significant sensitivity to vergence velocity.

\subsection{Saccades with the LEFT eye viewing (no eye
switching)}\label{saccades-with-the-left-eye-viewing-no-eye-switching}

\includegraphics{BootstrapSOAstrabismusReportPDF_files/figure-latex/saccades LEFT-1.pdf}
\includegraphics{BootstrapSOAstrabismusReportPDF_files/figure-latex/saccades LEFT-2.pdf}
\includegraphics{BootstrapSOAstrabismusReportPDF_files/figure-latex/saccades LEFT-3.pdf}

\subsubsection{Significance Summary}\label{significance-summary-2}

We analyzed 36 cells that had sufficiently many saccades with the left
eye viewing and found that 2 (6\%) had significant sensitivity to both
vergence velocity and vergence position. A total of 11 (31\%) cells
showed significant sensitivity to vergence angle during these saccades,
while 7 cells showed significant sensitivity to vergence velocity.

\subsection{Saccades with the RIGHT eye viewing (no eye
switching)}\label{saccades-with-the-right-eye-viewing-no-eye-switching}

\includegraphics{BootstrapSOAstrabismusReportPDF_files/figure-latex/saccades RIGHT-1.pdf}
\includegraphics{BootstrapSOAstrabismusReportPDF_files/figure-latex/saccades RIGHT-2.pdf}
\includegraphics{BootstrapSOAstrabismusReportPDF_files/figure-latex/saccades RIGHT-3.pdf}

\subsubsection{Significance Summary}\label{significance-summary-3}

We analyzed 45 cells that had sufficiently many saccades with the right
eye viewing and found that 1 (2\%) had significant sensitivity to both
vergence velocity and vergence position. A total of 5 (11\%) cells
showed significant sensitivity to vergence angle during these saccades,
while 7 cells showed significant sensitivity to vergence velocity.


\end{document}
